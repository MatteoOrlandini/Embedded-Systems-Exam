Descrizione algoritmo Global Memory