In algebra lineare, la decomposizione ai valori singolari (SVD), è una fattorizzazione di una matrice in tre diverse matrici basata sull'uso di autovalori e autovettori.

La decomposizione di una matrice $\mathbf{A}$ si basa sul \textit{teorema fondamentale dell'algebra lineare}: 

Data una matrice $\mathbf{A}\in\mathbb{C}^{m\times n}$ di rango $\rho$, la decomposizione in valori singolari di $\mathbf{A}$ è rappresentata dal prodotto di due matrici unitarie $\mathbf{U}\in\mathbb{C}^{m\times m}$, $\mathbf{V}\in\mathbb{C}^{n\times n}$ e una matrice diagonale $\mathbf{\Sigma}\in\mathbb{R}^{m\times n}$, come mostrato in (\ref{svd}).
\begin{equation}\label{svd}
\mathbf{A}=\mathbf{U}\mathbf{\Sigma}\mathbf{V}^H
\end{equation}
Le colonne di $\mathbf{U}$ sono chiamate \textit{vettori singolari sinistri} di $\mathbf{A}$ mentre le colonne di $\mathbf{V}$ sono i \textit{vettori singolari destri} di $\mathbf{A}$. Inoltre, $\mathbf{\Sigma}$ è una matrice reale non negativa del tipo
\[
\Sigma=\begin{bmatrix}
\sigma_1 & 0 & 0 & 0\\
0 & \ddots & 0 & 0\\
0 & 0 & \sigma_{\rho} & 0\\
0 & 0 & 0 & 0
\end{bmatrix}
\]
Gli elementi diagonali di $\Sigma$ sono i \textit{valori singolari} di $\mathbf{A}$ e sono solitamente in ordine decrescente: $\sigma_1 \geq \sigma_2 \geq ... \geq \sigma_{\rho} > 0, \sigma_{\rho+1}=...=0$.

Nel caso in cui $\mathbf{A}\in\mathbb{R}^{m\times n}$ l'equazione (\ref{svd}) diventa:
\begin{equation}\label{svd2}
\mathbf{A}=\mathbf{U}\mathbf{\Sigma}\mathbf{V}^T
\end{equation}
dove $\mathbf{U}\in\mathbb{R}^{m\times m}$ e $\mathbf{V}\in\mathbb{R}^{n\times n}$ sono matrici ortonormali.
\begin{eqnarray}
\mathbf{U}\mathbf{U}^T=\mathbf{U}^T\mathbf{U}=I^{m\times m}\\
\mathbf{V}\mathbf{V}^T=\mathbf{V}^T\mathbf{V}=I^{n\times n}
\end{eqnarray}
Unendo queste due proprietà all'equazione (\ref{svd2}) si possono ricavare l'espressione di $\mathbf{A}\mathbf{A}^T$:
\[
\mathbf{A}\mathbf{A}^T=\mathbf{U}\mathbf{\Sigma}\mathbf{V}^T(\mathbf{U}\mathbf{\Sigma}\mathbf{V}^T)^T=\mathbf{U}\mathbf{\Sigma}\mathbf{V}^T\mathbf{V}\mathbf{\Sigma}\mathbf{U}^T=\mathbf{U}\mathbf{\Sigma}^2\mathbf{U}^T
\]
Allo stesso modo, per $\mathbf{A}^T\mathbf{A}$:
\[
\mathbf{A}^T\mathbf{A}=\mathbf{V}\mathbf{\Sigma}^2\mathbf{V}^T
\]
La SVD ha numerose applicazioni nel campo dell'algebra lineare. Innanzitutto fornisce delle informazioni importanti sulla matrice $\mathbf{A}$, come il suo rango, qual è il suo nucleo e qual è la sua immagine. Viene usata per definire la pseudo-inversa di una matrice rettangolare utile per la risoluzione del problema dei minimi quadrati. Trova utilizzo anche nella risoluzione di sistema di equazioni lineari omogeneo.

Un'altra importante applicazione riguarda l'approssimazione della matrice $\mathbf{A}$, con una di rango inferiore (SVD troncata), utilizzata nell'elaborazione di immagini e nell'elaborazione dei segnali.

La SVD ha anche note applicazioni nel campo dell'analisi delle componenti principali.


