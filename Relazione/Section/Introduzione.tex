\label{sec:Introduzione}
La seguente tesina descrive il lavoro svolto per la progettazione di un algoritmo di One-sided Jacobi per la decomposizione in valori singolari (SVD) di una matrice, tramite librerie NVIDIA CUDA C. Il codice è stato testato e pensato per una piattaforma embedded Jetson TK1.

La Jetson TK1 è una scheda grafica embedded realizzato dalla NVIDIA che contiene un processore Tegra K1 SoC nella variante T124. Inoltre essa possiede un sistema operativo Ubuntu Linux ed ha installata la versione di CUDA 6.5. La compute capability della GPU, cioè la capacità di calcolo che determina le specifiche generali e le funzionalità disponibili, è pari a 3.2.

In facoltà era già stato sviluppato un codice in C per la SVD ottimizzato per lavorare su CPU. Il nostro lavoro era quello di confrontare i nostri risultati con quelli precedentemente ottenuti e valutare la possibilità di una implementazione alternativa che sfruttasse appieno le potenzialità del calcolo parallelo della GPU della Jetson.

Come base per il nostro lavoro abbiamo studiato il linguaggio di programmazione e l’architettura CUDA in~\cite{Cheng:ProfessionalCudaProgramming} e~\cite{Sanders:CudaByExample}. In seguito sono stati analizzati gli articoli~\cite{Acosta:SVD},~\cite{Boukaram:SVD},~\cite{Lahabar:SVD} e~\cite{Romer:SVD} che descrivevano vari approcci per l’ottimizzazione dell’algoritmo su GPU, per poi realizzare la nostra versione, i cui risultati sono molto simili a quelli ottenuti negli articoli. 

Per lavorare sulla scheda abbiamo sfruttato il protocollo SSH dal dipartimento, in modo da connettersi alla scheda da remoto. Alla fine della realizzazione del progetto non è però stato possibile accedere al dipartimento e quindi riportare i risultati esatti del tempo computazionale che necessita l’algoritmo sulla Jetson. Abbiamo quindi usato come riferimento una NVIDIA GTX 610M, i cui risultati sono proporzionali a quelli ottenuti con la
Jetson.